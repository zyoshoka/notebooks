\documentclass[a4paper]{ltjsarticle}

% font
\usepackage[haranoaji,deluxe,expert,bold,jis2004]{luatexja-preset}


% other extention for LuaTeX-ja
\usepackage{luatexja-adjust}
\ltjenableadjust[lineend=extended,priority=true]

\usepackage{luatexja-otf}


% fontspec
\ltjsetparameter{jacharrange={-2,-3}}
\usepackage{luatexja-fontspec}


% extended math fonts
\usepackage[full]{yhmath}


% replacing kutoten
\usepackage{newunicodechar}
\newunicodechar{、}{,}
\newunicodechar{。}{.}
\newcommand{\、}{\char"3001}
\newcommand{\。}{\char"3002}


% space after comma
% https://tex.stackexchange.com/questions/676840/take-space-after-comma-in-math-mode
\AtBeginDocument{%
  \mathchardef\stdcomma=\mathcode`,
  \mathcode`,="8000
}
\begingroup\lccode`~=`, \lowercase{\endgroup\def~}{\stdcomma\,}


% section style
\usepackage{titlesec}
\titleformat{\section}[block]
  {\normalfont}
  {\mdseries\thesection}
  {1\zw}
  {\titlerule\\\Large\bfseries}
\titleformat*{\subsection}{\large\bfseries}
\titleformat*{\subsubsection}{\bfseries}


% page style
\usepackage{fancyhdr}
\fancypagestyle{firstpage}{%
  \fancyhf{}
  \fancyhead[L]{}
  \fancyfoot[C]{―\quad\thepage\quad―}
  \renewcommand{\headrulewidth}{0pt}
}
\fancypagestyle{headings}{%
  \fancyhf{}
  \fancyhead[L]{\leftmark}
  \fancyfoot[C]{―\quad\thepage\quad―}
}
\pagestyle{headings}


% list style
\usepackage{enumitem}
\setlistdepth{1000}
\setlist[enumerate]{noitemsep,topsep=.5\zw}
\setlist[enumerate,1]{label={(\arabic*)},leftmargin=2\zw,labelsep=.75\zw,itemindent=1\zw,listparindent=1\zw}


% tcolorbox
\usepackage{tcolorbox}
\tcbuselibrary{breakable,skins,theorems}


% theorem style
\usepackage{amsthm}
\theoremstyle{definition}

\newtheorem{thm}{定理}[section]
\newtheorem{lem}[thm]{補題}
\newtheorem{prop}[thm]{命題}
\newtheorem{cor}[thm]{系}
\newtheorem{conj}[thm]{予想}
\newtheorem{dfn}[thm]{定義}
\newtheorem{rem}[thm]{注}
\newtheorem{exm}[thm]{例}
\tcbset{
  common/.style={
    enhanced,
    colframe=black, colback=white,
    coltitle=black, fonttitle=\bfseries, separator sign none,
    description color=black,
    sharp corners, frame hidden,
    boxsep=0mm, left=.9\zw, right=1\zw, middle=.5\zw,
    breakable
  },
  commonthm/.style={
    top=.5\zw, bottom=.5\zw, borderline={.1\zw}{0mm}{black,dotted}
  },
  commonproof/.style={
    top=0mm, bottom=0mm, borderline east={.5pt}{0mm}{black},
  }
}
\tcolorboxenvironment{thm}{common,commonthm}
\tcolorboxenvironment{lem}{common,commonthm}
\tcolorboxenvironment{prop}{common,commonthm}
\tcolorboxenvironment{cor}{common,commonthm}
\tcolorboxenvironment{conj}{common,commonthm}
\tcolorboxenvironment{dfn}{common,commonthm}
\tcolorboxenvironment{rem}{common,commonthm}
\tcolorboxenvironment{exm}{common,commonthm}


% proof style
\makeatletter
\renewenvironment{proof}[1][\proofname]{\par
  \pushQED{\qed}%
  \normalfont \topsep6\p@\@plus6\p@\relax
  \trivlist
  \item\relax
        {%\itshape
  #1。%\@addpunct{.}
  }\hspace\labelsep\ignorespaces
}{%
  \popQED\endtrivlist\@endpefalse
}
\makeatother

\tcolorboxenvironment{proof}{common,commonproof}
\renewcommand{\proofname}{\bfseries 証明}


% hypertext marks
\usepackage{hyperref}


% cleveref
\usepackage{cleveref}
\crefname{equation}{式}{式}
\crefname{enumi}{}{}


% URL
\usepackage{url}


% ipsum
\usepackage{lipsum}


% amsfonts
\usepackage{mathtools,amssymb}


% displaystyle
\newcommand{\sfrac}[2]{\genfrac{}{}{}{0}{\,#1\,}{\,#2\,}}
\newcommand{\dinf}{\displaystyle\inf}
\newcommand{\dint}{\displaystyle\int}
\newcommand{\dlim}{\displaystyle\lim}
\newcommand{\dmax}{\displaystyle\max}
\newcommand{\dmin}{\displaystyle\min}
\newcommand{\dprod}{\displaystyle\prod}
\newcommand{\dsum}{\displaystyle\sum}
\newcommand{\dsup}{\displaystyle\sup}


% roman abbrev
\newcommand{\GL}{\text{GL}}
\newcommand{\M}{\text{M}}
\newcommand{\SL}{\text{SL}}


% set-builder
% https://tex.stackexchange.com/questions/253077/how-do-you-create-a-set-in-latex
\DeclarePairedDelimiterX\set[1]\lbrace\rbrace{\def\given{\;\delimsize\vert\;}#1}


% array
% https://github.com/texjporg/jsclasses/issues/61
\newcommand{\nbl}{\narrowbaselines}
\everymath=\expandafter{\the\everymath \narrowbaselines}


% lineskip
\setlength{\lineskiplimit}{4pt}
\setlength{\lineskip}{4pt}


% generate abbreviations for mathbb, bm, mathcal, mathrsfs, and mathfrak
\usepackage{bm}
\usepackage{mathrsfs}
\directlua{
  local command_mapping = {
    ["bb"] = "mathbb",
    ["bm"] = "bm",
    ["cal"] = "mathcal",
    ["scr"] = "mathscr",
    ["frk"] = "mathfrak",
  }

  for command, replacement in pairs(command_mapping) do
    for charCode = string.byte('a'), string.byte('z') do
      local char = command == "bm" and string.char(charCode) or string.char(charCode):upper()
      tex.print("\\newcommand{\\" .. command .. char:lower() .. "}{\\" .. replacement .. "{" .. char .. "}}")
    end
  end
}


\begin{document}

\title{\texttt{notebooks} リポジトリを支える技術}
\author{zyoshoka\thanks{\url{https://zyoshoka.com}}\\\texttt{root@zyoshoka.com}}
\maketitle
\thispagestyle{firstpage}

この \texttt{notebooks} リポジトリ(\url{https://github.com/zyoshoka/notebooks})をセットアップするにあたってのメモをここにまとめています。私は\LaTeX をはじめて2年ほどですので、改善点などございましたらリポジトリのIssueないしPull Requestでご指南いただけますと幸いです。

\section{書式}

\subsection{メイン}
ドキュメントクラスに \texttt{ltjsarticle} を使用します。用紙はA4判です。メインファイルのプリアンブルに \texttt{\string\input\{\_preamble\}} と入力して \texttt{\_preamble.tex} を読み込んでおきます。要は使い回しを簡単にするためにこうしているわけです。

ただ、後述の \texttt{fancyhdr} パッケージで設定したページスタイルを最初のページに反映させるため、\texttt{\string\maketitle} のあとに \texttt{\string\thispagestyle\{firstpage\}} をいちいち挿入しています。ダサいのでより良い方法があれば是非教えてください。

\subsection{プリアンブル}
\texttt{\_preamble.tex} に書いている内容の説明を順番にしていきます。

Lua\LaTeX-jaの拡張パッケージをいくつか読み込んでいます。フォントは \texttt{luatexja-preset} パッケージで原ノ味フォントを使用するように設定しています。また、行長の補正に \texttt{luatexja-adjust} パッケージを読み込んでいます。また \texttt{otf}、\texttt{fontspec} パッケージに対応するパッケージとしてそれぞれ \texttt{luatexja-otf}、\texttt{luatexja-fontspec} パッケージを読み込んであります。

句読点のソース・ファイルへの入力は「\、\。」で行いますが、出力では「、。」となるように設定されています(IMEの設定を変更したくないためこのようにしています)。どうしても元の姿で出力する必要がある場合には \texttt{\string\、\string\。} のように入力します。

セクションスタイルは \texttt{titlesec} パッケージで少しイカした感じにしてあります。ページスタイルは \texttt{fancyhdr} パッケージで設定しています(なお、このリポジトリに属する文書は全て片面印刷向けのものであるとします)。箇条書きのスタイルは \texttt{enumitem} パッケージで設定しています。以下のように出力されます。
\begin{enumerate}
  \item あいうえおあいうえおあいうえおあいうえおあいうえおあいうえおあいうえおあいうえおあいうえおあいうえおあいうえおあいうえおあいうえお

  あいうえおあいうえおあいうえおあいうえお
  \item かきくけこ
\end{enumerate}

定理環境に \texttt{amsthm} パッケージを用いますが、枠がほしいので \texttt{tcolorbox} パッケージも併用しています。この他には、補題、命題、系、予想、定義、注、例を用意しています。
\begin{thm}[ねこちゃんの定理]
  すごい定理。あああああああああああああああああああああああああああああああああああああああああああああああああああああああああああああああああああああああ
  \begin{enumerate}
    \item ねこはかわいい。
    \item とりはねこよりかわいい。
  \end{enumerate}
\end{thm}
\begin{proof}
  略。
\end{proof}

行列。括弧の丸みがかわいいですね。
\begin{equation}
  {\begin{pmatrix}
    1 & 2 & 3
  \end{pmatrix}}{\begin{pmatrix}
    4 \\ 5 \\ 6
  \end{pmatrix}} = 32
\end{equation}

集合。\texttt{\string\set*\{x\string\given\string\exists\ n\string\in\string\bbn, x = 2n\}} みたいに書くと $\set*{x\given\exists n\in\bbn, x = 2n}$ となります。何が嬉しいかというと
\begin{equation}
  \set*{(r, \theta)\given 0\leq r\leq 1, 0\leq\theta\leq\sfrac{\pi}{2}}
\end{equation}
のように括弧と縦棒のサイズが中身に応じて自動で調整されるようになっているわけです。

コマンド系。$\bbr$ は \texttt{\string\bbr}、$\bmv$ は \texttt{\string\bmv}、$\calo$ は \texttt{\string\calo}、$\scrc$ は \texttt{\string\scrc}、$\frks$ は $\texttt{\string\frks}$ といった具合で出るようになっています。

他にも複数のパッケージを読み込んで複数のコマンドを定義してありますが、それらについては特筆すべきことがないので省略します。

\section{その他}

\subsection{Git}
ビルド成果物は全てトラッキングされないように設定しています。なお、\texttt{.pdf} ファイルはGitHub Actionsによって生成されており \texttt{pdf} ブランチから閲覧できます。

\subsection{GitHub Actions}
\texttt{main} ブランチに \texttt{push} されたとき実行されるようになっています。具体的には、\texttt{\_} で始まるファイル名のものを除く全ての \texttt{.tex} ファイルについて \texttt{.latexmkrc} ファイルでの記述に基づいたビルドが実行され、生成された全ての \texttt{.pdf} ファイルだけが(ディレクトリ構造を保ったまま)\texttt{pdf} ブランチへコミットされます。

\end{document}
