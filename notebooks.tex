\documentclass[a4paper]{ltjsarticle}

% font
\usepackage[haranoaji,deluxe,expert,bold,jis2004]{luatexja-preset}


% Lua scripting
\usepackage{luacode}


% other extension for LuaTeX-ja
\usepackage{luatexja-adjust}
\ltjenableadjust[lineend=extended,priority=true]

\usepackage{luatexja-otf}


% fontspec
\ltjsetparameter{jacharrange={-2,-3}}
\usepackage{luatexja-fontspec}


% extended math fonts
\usepackage[full]{yhmath}


% replacing kutoten
\usepackage{newunicodechar}
\newunicodechar{、}{,}
\newunicodechar{。}{.}
\newcommand{\、}{\char"3001}
\newcommand{\。}{\char"3002}


% space after comma
% https://tex.stackexchange.com/questions/676840/take-space-after-comma-in-math-mode
\AtBeginDocument{%
  \mathchardef\stdcomma=\mathcode`,
  \mathcode`,="8000
}
\begingroup\lccode`~=`, \lowercase{\endgroup\def~}{\stdcomma\,}


% section style
\usepackage{titlesec}
\titleformat{\section}[block]
  {\normalfont}
  {\mdseries\thesection}
  {1\zw}
  {\titlerule\\\Large\bfseries}
\titleformat*{\subsection}{\large\bfseries}
\titleformat*{\subsubsection}{\bfseries}


% page style
\usepackage{fancyhdr}
\fancypagestyle{firstpage}{%
  \fancyhf{}
  \fancyhead[L]{}
  \fancyfoot[C]{―\quad\thepage\quad―}
  \renewcommand{\headrulewidth}{0pt}
}
\fancypagestyle{headings}{%
  \fancyhf{}
  \fancyhead[L]{\leftmark}
  \fancyfoot[C]{―\quad\thepage\quad―}
}
\pagestyle{headings}


% list style
\usepackage{enumitem}
\setlistdepth{1000}
\setlist[enumerate]{noitemsep,topsep=.5\zw}
\setlist[enumerate,1]{label={(\arabic*)},leftmargin=2\zw,labelsep=.75\zw,itemindent=1\zw,listparindent=1\zw}
\setlist[itemize]{noitemsep,topsep=.5\zw}


% tcolorbox
\usepackage{tcolorbox}
\tcbuselibrary{breakable,skins,theorems}


% theorem style
\usepackage{amsthm}
\theoremstyle{definition}

\newtheorem{thm}{定理}[section]
\newtheorem{lem}[thm]{補題}
\newtheorem{prop}[thm]{命題}
\newtheorem{cor}[thm]{系}
\newtheorem{conj}[thm]{予想}
\newtheorem{dfn}[thm]{定義}
\newtheorem{rem}[thm]{注}
\newtheorem{exm}[thm]{例}
\newtheorem{mondai}{問題}
\newtheorem*{mondai*}{問題}
\newtheorem{wakunasimondai}{問題}
\newtheorem*{wakunasimondai*}{問題}
\tcbset{
  common/.style={
    enhanced,
    colframe=black, colback=white,
    coltitle=black, fonttitle=\bfseries, separator sign none,
    description color=black,
    sharp corners, frame hidden,
    boxsep=0mm, left=.9\zw, right=1\zw, middle=.5\zw,
    before upper={\parindent=1\zw},
    breakable
  },
  commonthm/.style={
    top=.5\zw, bottom=.5\zw, borderline={.1\zw}{0mm}{black,dotted}
  },
  commonproof/.style={
    top=0mm, bottom=0mm, borderline east={.5pt}{0mm}{black},
  }
}
\tcolorboxenvironment{thm}{common,commonthm}
\tcolorboxenvironment{lem}{common,commonthm}
\tcolorboxenvironment{prop}{common,commonthm}
\tcolorboxenvironment{cor}{common,commonthm}
\tcolorboxenvironment{conj}{common,commonthm}
\tcolorboxenvironment{dfn}{common,commonthm}
\tcolorboxenvironment{rem}{common,commonthm}
\tcolorboxenvironment{exm}{common,commonthm}
\tcolorboxenvironment{mondai}{common,commonthm}
\tcolorboxenvironment{mondai*}{common,commonthm}
% \tcolorboxenvironment{wakunasimondai}{common}


% proof style
% \makeatletter
% \renewenvironment{proof}[1][\proofname]{\par
%   \pushQED{\qed}%
%   \normalfont \topsep6\p@\@plus6\p@\relax
%   \trivlist
%   \item\relax
%         {%\itshape
%   #1。%\@addpunct{.}
%   }\hspace\labelsep\ignorespaces
% }{%
%   \popQED\endtrivlist\@endpefalse
% }
% \makeatother

\tcolorboxenvironment{proof}{common,commonproof}
\renewcommand{\proofname}{\bfseries 証明}


% makes table useful
\usepackage{multirow}
\usepackage{threeparttable}
\usepackage{booktabs}
\usepackage{csvsimple}
\usepackage{supertabular}


% hypertext marks
\usepackage{hyperref}


% cleveref
\usepackage[noabbrev]{cleveref}
\begin{luacode*}
  -- \verb|
  local cref_mapping = {
    ["appendix"] = "付録",
    ["equation"] = "式",
    ["enumi"] = "",
    ["figure"] = "図",
    ["section"] = "節",
    ["table"] = "表",
    ["listing"] = "ソースコード",
    ["wakunasimondai"] = "問題",
  }

  for cref, label in pairs(cref_mapping) do
    tex.sprint("\\crefname{" .. cref .. "}{" .. label .. "}{" .. label .. "}")
    tex.sprint("\\crefformat{" .. cref .. "}{" .. label .. "#2#1#3}")
  end -- |
\end{luacode*}
\newcommand{\crefrangeconjunction}{\horizontalbar}
\newcommand{\crefpairconjunction}{、}
\newcommand{\crefmiddleconjunction}{、}
\newcommand{\creflastconjunction}{、}


% URL
\usepackage{url}


% BibLaTeX
\usepackage{biblatex}


% tategaki
% https://qiita.com/zr_tex8r/items/6f0b88c5838c42241457
\usepackage{lltjext}


% ipsum
\usepackage{lipsum}


% amsfonts
\usepackage{mathtools,amssymb}


% SI units
\usepackage{siunitx}
\sisetup{
  separate-uncertainty,
  per-mode=symbol,
  input-digits = 0123456789\pi,
  uncertainty-mode = separate,
  output-open-uncertainty = [,
  output-close-uncertainty = ],
  uncertainty-separator = \,,
  separate-uncertainty-units = bracket,
}


% multi columns
\usepackage{multicol}


% TikZ
\usepackage{tikz}
\usetikzlibrary{arrows.meta,petri,positioning}
\usepackage{standalone}
\usepackage{pgfplots}
\pgfplotsset{compat=1.18}
\usepackage{pgfplotstable}
\usepackage{shellesc}
\usepackage{gnuplottex}
\usepackage{gnuplot-lua-tikz}

% https://tex.stackexchange.com/questions/36147/how-do-i-get-the-gnuplottex-epstopdf-package-to-work-with-output-dir-and-au
\usepackage{etoolbox}
\let\nodirfigname\figname
\def\figname{./out/\nodirfigname}
\expandafter\patchcmd\csname\string\gnuplot\endcsname
  {\figname}{\nodirfigname}{}{}


% wrapfig
\usepackage{wrapfig}

% https://tex.stackexchange.com/questions/111393/too-much-space-around-wrap-figure
\setlength{\intextsep}{0pt}
\setlength{\columnsep}{0pt}


% subfigure
\usepackage{subcaption}


% define minipage environment with an unknown width
% use case: https://stackoverflow.com/questions/59551243/how-to-align-an-enumerated-list-in-latex
\usepackage{varwidth}
\newenvironment{centered}{
  \begin{center}
    \begin{varwidth}{\textwidth}
}{
    \end{varwidth}
  \end{center}
}


% multicolumn itemize
% reduce margin of the environment
\newenvironment{multicolsitemize}[1]{
  \vspace*{-.5\multicolsep}
  \begin{multicols}{#1}
    \begin{itemize}
}{
    \end{itemize}
  \end{multicols}
  \vspace*{-.5\multicolsep}
}


% tweaking vertical space
\newcommand{\tweak}[1]{\vspace{#1\zw}}


% kintou
\newcommand{\kintou}[2]{\texorpdfstring{\kintouwidth{#1\zw}{#2}}{#2}}
% 美文書作成入門に載ってるもの
% fancyhdr の中に入るとうまくいかないので使えないのてとりあえず下で我慢
% \newcommand{\kintouwidth}[2]{\leavevmode\hbox to #1{%
%   \ltjsetparameter{kanjiskip=0pt plus 1fill minus 1fill}%
%   \ltjsetparameter{xkanjiskip=\ltjgetparameter{kanjiskip}}%
%   #2}}
% \makeatletter
\newcommand{\kintouwidth}[2]{\makebox[#1][s]{#2}}


% mask
% https://tex.stackexchange.com/questions/548426/how-to-make-text-appear-in-the-middle-of-a-phantom-in-math-mode
\usepackage{calc}
\newcommand*{\mask}[2]{%
  \makebox[\widthof{\(#1\)}]{\(#2\)}%
}


% displaystyle
\newcommand{\sfrac}[2]{\genfrac{}{}{}{0}{\,#1\,}{\,#2\,}}
\newcommand{\dinf}{\displaystyle\inf}
\newcommand{\dint}{\displaystyle\int}
\newcommand{\dlim}{\displaystyle\lim}
\newcommand{\dmax}{\displaystyle\max}
\newcommand{\dmin}{\displaystyle\min}
\newcommand{\dprod}{\displaystyle\prod}
\newcommand{\dsum}{\displaystyle\sum}
\newcommand{\dsup}{\displaystyle\sup}


% horizontal bar
\newcommand{\horizontalbar}{\symbol{8213}}


% roman abbrev
\newcommand{\GL}{\text{GL}}
\newcommand{\M}{\text{M}}
\newcommand{\SL}{\text{SL}}


% op
\DeclareMathOperator{\ch}{char}
\DeclareMathOperator{\id}{id}


% set-builder
% https://tex.stackexchange.com/questions/253077/how-do-you-create-a-set-in-latex
\DeclarePairedDelimiterX\set[1]\lbrace\rbrace{\def\given{\;\delimsize\vert\;}#1}


% round function
% https://tex.stackexchange.com/questions/433101/rounding-to-nearest-integer-symbol-in-latex
\DeclarePairedDelimiter{\nint}{\lfloor}{\rceil}


% array
% https://github.com/texjporg/jsclasses/issues/61
\newcommand{\nbl}{\narrowbaselines}
\everymath=\expandafter{\the\everymath \narrowbaselines}


% lineskip
\setlength{\lineskiplimit}{4pt}
\setlength{\lineskip}{4pt}


% generate abbreviations for mathbb, bm, mathcal, mathrsfs, and mathfrak
\usepackage{bm}
\usepackage{mathrsfs}
\directlua{
  local command_mapping = {
    ["bb"] = "mathbb",
    ["bm"] = "bm",
    ["cal"] = "mathcal",
    ["rm"] = "mathrm",
    ["scr"] = "mathscr",
    ["frk"] = "mathfrak",
  }

  for command, replacement in pairs(command_mapping) do
    for charCode = string.byte('a'), string.byte('z') do
      local char = command == "bm" and string.char(charCode) or string.char(charCode):upper()
      tex.print("\\newcommand{\\" .. command .. char:lower() .. "}{\\" .. replacement .. "{" .. char .. "}}")
    end
  end
}


% https://tex.stackexchange.com/a/548869
\usepackage{ifthen}
\usepackage{currfile-abspath}
\getabspath{\jobname.tex}
\ifthenelse{\equal{\theabsdir}{\thepwd}}% using ifthen package
%\ifdefstrequal{\theabsdir}{\thepwd}% using etoolbox package
    {\PassOptionsToPackage{outputdir={out}}{minted}}{\PassOptionsToPackage{outputdir={\theabsdir/out}}{minted}}


% see for setup
% https://zenn.dev/nac_39/scraps/25ffb7d0747a9d
% \directlua の前に置くと何故かエラーを吐く
\usepackage{minted}
\definecolor{bg}{rgb}{0.97,0.97,0.97}
\setminted{
  baselinestretch=.75,
  bgcolor=bg,
  breaklines,
  breakanywhere,
  gobble=2,
}
\renewcommand{\listingscaption}{ソースコード}
\usepackage{caption}
\captionsetup[listing]{labelsep=quad}
\newenvironment{longlisting}{\captionsetup{type=listing}}{}

% prevent caption 1zw sep overwritten
% (maybe overwritten by `subcaption` package?)
% @see texmf-dist/tex/luatex/luatexja/ltjsarticle.cls
\makeatletter
\long\def\@makecaption#1#2{{\small
  \advance\leftskip .0628\linewidth
  \advance\rightskip .0628\linewidth
  \vskip\abovecaptionskip
  \sbox\@tempboxa{#1{\hskip1\zw}#2}%
  \ifdim \wd\@tempboxa <\hsize \centering \fi
  #1{\hskip1\zw}#2\par
  \vskip\belowcaptionskip}}
\makeatother

\addbibresource{refs.bib}

\begin{document}

\title{\texttt{notebooks} リポジトリを支える技術}
\author{zyoshoka\thanks{\url{https://zyoshoka.com}}}
\maketitle
\thispagestyle{firstpage}

この \texttt{notebooks} リポジトリ(\url{https://github.com/zyoshoka/notebooks})をセットアップするにあたってのメモをここにまとめています。

\section{書式}

\subsection{メイン}
ドキュメントクラスに \texttt{ltjsarticle} を使用します。用紙はA4判です。メインファイルのプリアンブルに \texttt{\string\input\{\_preamble\}} と入力して \texttt{\_preamble.tex} を読み込んでおきます。要は使い回しを簡単にするためにこうしているわけです。

ただ、後述の \texttt{fancyhdr} パッケージで設定したページスタイルを最初のページに反映させるため、\texttt{\string\maketitle} のあとに \texttt{\string\thispagestyle\{firstpage\}} をいちいち挿入しています。ダサいのでより良い方法があれば是非教えてください。

\subsection{プリアンブル}
\texttt{\_preamble.tex} に書いている内容の説明を順番にしていきます。

Lua\LaTeX-jaの拡張パッケージをいくつか読み込んでいます。フォントは \texttt{luatexja-preset} パッケージで原ノ味フォントを使用するように設定しています。また、行長の補正に \texttt{luatexja-adjust} パッケージを読み込んでいます。また \texttt{otf}、\texttt{fontspec} パッケージに対応するパッケージとしてそれぞれ \texttt{luatexja-otf}、\texttt{luatexja-fontspec} パッケージを読み込んであります。

句読点のソース・ファイルへの入力は「\、\。」で行いますが、出力では「、。」となるように設定されています(IMEの設定を変更したくないためこのようにしています)。どうしても元の姿で出力する必要がある場合には \texttt{\string\、\string\。} のように入力します。

セクションスタイルは \texttt{titlesec} パッケージで少しイカした感じにしてあります。ページスタイルは \texttt{fancyhdr} パッケージで設定しています(なお、このリポジトリに属する文書は全て片面印刷向けのものであるとします)。箇条書きのスタイルは \texttt{enumitem} パッケージで設定しています。以下のように出力されます。
\begin{enumerate}
  \item あいうえおあいうえおあいうえおあいうえおあいうえおあいうえおあいうえおあいうえおあいうえおあいうえおあいうえおあいうえおあいうえお

  あいうえおあいうえおあいうえおあいうえお
  \item かきくけこ
\end{enumerate}

定理環境に \texttt{amsthm} パッケージを用いますが、枠がほしいので \texttt{tcolorbox} パッケージも併用しています。この他には、補題、命題、系、予想、定義、注、例を用意しています。
\begin{thm}[ねこちゃんの定理]
  すごい定理。あああああああああああああああああああああああああああああああああああああああああああああああああああああああああああああああああああああああ
  \begin{enumerate}
    \item ねこはかわいい。
    \item とりはねこよりかわいい。
  \end{enumerate}
\end{thm}
\begin{proof}
  略。
\end{proof}

行列。\cref{eq:matrix}。
\begin{equation}\label{eq:matrix}
  {\begin{pmatrix}
    1 & 2 & 3
  \end{pmatrix}}{\begin{pmatrix}
    4 \\ 5 \\ 6
  \end{pmatrix}} = 32
\end{equation}

集合。\texttt{\string\set*\{x\string\given\string\exists\ n\string\in\string\bbn, x = 2n\}} みたいに書くと $\set*{x\given\exists n\in\bbn, x = 2n}$ となります。何が嬉しいかというと
\begin{equation}
  \set*{(r, \theta)\given 0\leq r\leq 1, 0\leq\theta\leq\sfrac{\pi}{2}}
\end{equation}
のように括弧と縦棒のサイズが中身に応じて自動で調整されるようになっているわけです。

コマンド系。$\bbr$ は \texttt{\string\bbr}、$\bmv$ は \texttt{\string\bmv}、$\calo$ は \texttt{\string\calo}、$\scrc$ は \texttt{\string\scrc}、$\frks$ は $\texttt{\string\frks}$ といった具合で出るようになっています。

Ti\textit{k}Zが使えます。\cref{fig:pb-sn-phase-diagram}にTi\textit{k}Zで書いたPb--Sn合金の状態図を示します。\cref{fig:3d-boxes}のようにgnuplotからのトランスパイルでも使えます。
\begin{figure}[H]
  \centering
  \includestandalone[
    mode=image,
    extension=.pdf,
    scale=.9
  ]{tikz/out/pb-sn-phase-diagram}
  \caption{Pb--Sn合金の状態図}
  \label{fig:pb-sn-phase-diagram}
\end{figure}

\begin{figure}[t]
\centering
\begin{gnuplot}[terminal=tikz, terminaloptions={plotsize 9cm, 6cm}]
# Hack to parse as plain text for textlint
#\verb|
set boxwidth 0.4 absolute
set boxdepth 0.3
set style fill solid 1.00 border lt -1
set grid nopolar
set grid xtics nomxtics ytics nomytics ztics nomztics nortics nomrtics nox2tics nomx2tics noy2tics nomy2tics nocbtics nomcbtics
set grid vertical layerdefault lt 0 linecolor 0 linewidth 1.000, lt 0 linecolor 0 linewidth 1.000
unset key
set wall z0 fc rgb "slategrey" fillstyle transparent solid 0.50 border lt -1
set view 59, 24, 1, 1
set style data lines
set xyplane at 0
set xrange [ * : * ] noreverse writeback
set x2range [ * : * ] noreverse writeback
set yrange [ 0.00000 : 6.00000 ] noreverse nowriteback
set y2range [ * : * ] noreverse writeback
set zrange [ * : * ] noreverse writeback
set cbrange [ * : * ] noreverse writeback
set rrange [ * : * ] noreverse writeback
set pm3d depthorder base
set pm3d interpolate 1,1 flush begin noftriangles border lt black linewidth 1.000 dashtype solid corners2color mean
set pm3d lighting primary 0.5 specular 0.2 spec2 0
set colorbox vertical origin screen 0.9, 0.2 size screen 0.05, 0.6 front noinvert bdefault
rgbfudge(x) = x*51*32768 + (11-x)*51*128 + int(abs(5.5-x)*510/9.)
ti(col) = sprintf("%d",col)
NO_ANIMATION = 1
$data << EOD
1 1.5 2 2.4 4 6
2 1.5 3 3.5 4 5.5
3 4.5 5 5.5 6 6.5
4 3.7 4.5 5.0 5.5 6.1
5 3.1 3.5 4.2 5 6.1
6 1 4 5.0 6 9
7 4 4 4.8 6 6.1
8 4 5 5.1 6 6.1
9 1.5 2 2.4 3 3.5
10 2.7 3 3.5 4 4.3
EOD
splot for [col=1:5] $data using 1:(col):(col*column(col)):(rgbfudge($1)) with boxes fc rgb variable#|
\end{gnuplot}
\caption{Full treatment: 3D boxes with pm3d depth sorting and lighting~\cite{gnuplot-boxes3d}}
\label{fig:3d-boxes}
\end{figure}

ソースコード。\cref{listing:hello-world-js}にHello world!を出力するJavaScriptコードを示します。
\begin{longlisting}
  \begin{minted}[gobble=2]{js}
    (function () {
      console.log([
        72, 101, 108, 108, 111, 32, 119, 111, 114, 108, 100, 33
      ].map(function (c) {
        return String.fromCharCode(c)
      }).join(''));
    })();
  \end{minted}
  \caption{Hello world!を出力するJavaScriptコード(冗長)}
  \label{listing:hello-world-js}
\end{longlisting}

他にも複数のパッケージを読み込んで複数のコマンドを定義してありますが、それらについては特筆すべきことがないので省略します。

\section{その他}

\subsection{Git}
ビルド成果物は全てトラッキングされないように設定しています。なお、\texttt{.pdf} ファイルはGitHub Actionsによって生成されており \texttt{pdf} ブランチから閲覧できます。

\subsection{GitHub Actions}
\texttt{main} ブランチに \texttt{push} されたときとPull Requestを受け取ったとき(正確にはその上でレビューがリクエストされたとき)実行されるようになっています。具体的には、\texttt{\_} で始まるファイル名のものを除く全ての \texttt{.tex} ファイルについて \texttt{.latexmkrc} ファイルでの記述に基づいたビルドが実行され、生成された全ての \texttt{.pdf} ファイルだけが(ディレクトリ構造を保ったまま)\texttt{pdf} ブランチへコミットされます。

\printbibliography[title={参考文献}, heading=bibnumbered]

\end{document}
