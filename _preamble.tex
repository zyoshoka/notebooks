% font
\usepackage[haranoaji,deluxe,expert,bold,jis2004]{luatexja-preset}


% other extention for LuaTeX-ja
\usepackage{luatexja-adjust}
\ltjenableadjust[lineend=extended,priority=true]

\usepackage{luatexja-otf}


% fontspec
\ltjsetparameter{jacharrange={-2,-3}}
\usepackage{luatexja-fontspec}


% extended math fonts
\usepackage[full]{yhmath}


% replacing kutoten
\usepackage{newunicodechar}
\newunicodechar{、}{,}
\newunicodechar{。}{.}
\newcommand{\、}{\char"3001}
\newcommand{\。}{\char"3002}


% space after comma
% https://tex.stackexchange.com/questions/676840/take-space-after-comma-in-math-mode
\AtBeginDocument{%
  \mathchardef\stdcomma=\mathcode`,
  \mathcode`,="8000
}
\begingroup\lccode`~=`, \lowercase{\endgroup\def~}{\stdcomma\,}


% section style
\usepackage{titlesec}
\titleformat{\section}[block]
  {\normalfont}
  {\mdseries\thesection}
  {1\zw}
  {\titlerule\\\Large\bfseries}
\titleformat*{\subsection}{\large\bfseries}
\titleformat*{\subsubsection}{\bfseries}


% page style
\usepackage{fancyhdr}
\fancypagestyle{firstpage}{%
  \fancyhf{}
  \fancyhead[L]{}
  \fancyfoot[C]{―\quad\thepage\quad―}
  \renewcommand{\headrulewidth}{0pt}
}
\fancypagestyle{headings}{%
  \fancyhf{}
  \fancyhead[L]{\leftmark}
  \fancyfoot[C]{―\quad\thepage\quad―}
}
\pagestyle{headings}


% list style
\usepackage{enumitem}
\setlistdepth{1000}
\setlist[enumerate]{noitemsep,topsep=.5\zw}
\setlist[enumerate,1]{label={(\arabic*)},leftmargin=2\zw,labelsep=.75\zw,itemindent=1\zw,listparindent=1\zw}


% tcolorbox
\usepackage{tcolorbox}
\tcbuselibrary{breakable,skins,theorems}


% theorem style
\usepackage{amsthm}
\theoremstyle{definition}

\newtheorem{thm}{定理}[section]
\newtheorem{lem}[thm]{補題}
\newtheorem{prop}[thm]{命題}
\newtheorem{cor}[thm]{系}
\newtheorem{conj}[thm]{予想}
\newtheorem{dfn}[thm]{定義}
\newtheorem{rem}[thm]{注}
\newtheorem{exm}[thm]{例}
\tcbset{
  common/.style={
    enhanced,
    colframe=black, colback=white,
    coltitle=black, fonttitle=\bfseries, separator sign none,
    description color=black,
    sharp corners, frame hidden,
    boxsep=0mm, left=.9\zw, right=1\zw, middle=.5\zw,
    breakable
  },
  commonthm/.style={
    top=.5\zw, bottom=.5\zw, borderline={.1\zw}{0mm}{black,dotted}
  },
  commonproof/.style={
    top=0mm, bottom=0mm, borderline east={.5pt}{0mm}{black},
  }
}
\tcolorboxenvironment{thm}{common,commonthm}
\tcolorboxenvironment{lem}{common,commonthm}
\tcolorboxenvironment{prop}{common,commonthm}
\tcolorboxenvironment{cor}{common,commonthm}
\tcolorboxenvironment{conj}{common,commonthm}
\tcolorboxenvironment{dfn}{common,commonthm}
\tcolorboxenvironment{rem}{common,commonthm}
\tcolorboxenvironment{exm}{common,commonthm}


% proof style
\makeatletter
\renewenvironment{proof}[1][\proofname]{\par
  \pushQED{\qed}%
  \normalfont \topsep6\p@\@plus6\p@\relax
  \trivlist
  \item\relax
        {%\itshape
  #1。%\@addpunct{.}
  }\hspace\labelsep\ignorespaces
}{%
  \popQED\endtrivlist\@endpefalse
}
\makeatother

\tcolorboxenvironment{proof}{common,commonproof}
\renewcommand{\proofname}{\bfseries 証明}


% hypertext marks
\usepackage{hyperref}


% cleveref
\usepackage{cleveref}
\crefname{equation}{式}{式}
\crefname{enumi}{}{}


% URL
\usepackage{url}


% ipsum
\usepackage{lipsum}


% amsfonts
\usepackage{mathtools,amssymb}


% displaystyle
\newcommand{\sfrac}[2]{\genfrac{}{}{}{0}{\,#1\,}{\,#2\,}}
\newcommand{\dinf}{\displaystyle\inf}
\newcommand{\dint}{\displaystyle\int}
\newcommand{\dlim}{\displaystyle\lim}
\newcommand{\dmax}{\displaystyle\max}
\newcommand{\dmin}{\displaystyle\min}
\newcommand{\dprod}{\displaystyle\prod}
\newcommand{\dsum}{\displaystyle\sum}
\newcommand{\dsup}{\displaystyle\sup}


% roman abbrev
\newcommand{\GL}{\text{GL}}
\newcommand{\M}{\text{M}}
\newcommand{\SL}{\text{SL}}


% set-builder
% https://tex.stackexchange.com/questions/253077/how-do-you-create-a-set-in-latex
\DeclarePairedDelimiterX\set[1]\lbrace\rbrace{\def\given{\;\delimsize\vert\;}#1}


% array
% https://github.com/texjporg/jsclasses/issues/61
\newcommand{\nbl}{\narrowbaselines}
\everymath=\expandafter{\the\everymath \narrowbaselines}


% lineskip
\setlength{\lineskiplimit}{4pt}
\setlength{\lineskip}{4pt}


% generate abbreviations for mathbb, bm, mathcal, mathrsfs, and mathfrak
\usepackage{bm}
\usepackage{mathrsfs}
\directlua{
  local command_mapping = {
    ["bb"] = "mathbb",
    ["bm"] = "bm",
    ["cal"] = "mathcal",
    ["scr"] = "mathscr",
    ["frk"] = "mathfrak",
  }

  for command, replacement in pairs(command_mapping) do
    for charCode = string.byte('a'), string.byte('z') do
      local char = command == "bm" and string.char(charCode) or string.char(charCode):upper()
      tex.print("\\newcommand{\\" .. command .. char:lower() .. "}{\\" .. replacement .. "{" .. char .. "}}")
    end
  end
}
