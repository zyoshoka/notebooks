% font
\usepackage[haranoaji,deluxe,expert,bold,jis2004]{luatexja-preset}


% other extension for LuaTeX-ja
\usepackage{luatexja-adjust}
\ltjenableadjust[lineend=extended,priority=true]

\usepackage{luatexja-otf}


% fontspec
\ltjsetparameter{jacharrange={-2,-3}}
\usepackage{luatexja-fontspec}


% extended math fonts
\usepackage[full]{yhmath}


% replacing kutoten
\usepackage{newunicodechar}
\newunicodechar{、}{,}
\newunicodechar{。}{.}
\newcommand{\、}{\char"3001}
\newcommand{\。}{\char"3002}


% space after comma
% https://tex.stackexchange.com/questions/676840/take-space-after-comma-in-math-mode
\AtBeginDocument{%
  \mathchardef\stdcomma=\mathcode`,
  \mathcode`,="8000
}
\begingroup\lccode`~=`, \lowercase{\endgroup\def~}{\stdcomma\,}


% section style
\usepackage{titlesec}
\titleformat{\section}[block]
  {\normalfont}
  {\mdseries\thesection}
  {1\zw}
  {\titlerule\\\Large\bfseries}
\titleformat*{\subsection}{\large\bfseries}
\titleformat*{\subsubsection}{\bfseries}


% page style
\usepackage{fancyhdr}
\fancypagestyle{firstpage}{%
  \fancyhf{}
  \fancyhead[L]{}
  \fancyfoot[C]{―\quad\thepage\quad―}
  \renewcommand{\headrulewidth}{0pt}
}
\fancypagestyle{headings}{%
  \fancyhf{}
  \fancyhead[L]{\leftmark}
  \fancyfoot[C]{―\quad\thepage\quad―}
}
\pagestyle{headings}


% list style
\usepackage{enumitem}
\setlistdepth{1000}
\setlist[enumerate]{noitemsep,topsep=.5\zw}
\setlist[enumerate,1]{label={(\arabic*)},leftmargin=2\zw,labelsep=.75\zw,itemindent=1\zw,listparindent=1\zw}
\setlist[itemize]{noitemsep,topsep=.5\zw}


% tcolorbox
\usepackage{tcolorbox}
\tcbuselibrary{breakable,skins,theorems}


% theorem style
\usepackage{amsthm}
\theoremstyle{definition}

\newtheorem{thm}{定理}[section]
\newtheorem{lem}[thm]{補題}
\newtheorem{prop}[thm]{命題}
\newtheorem{cor}[thm]{系}
\newtheorem{conj}[thm]{予想}
\newtheorem{dfn}[thm]{定義}
\newtheorem{rem}[thm]{注}
\newtheorem{exm}[thm]{例}
\newtheorem{mondai}{問題}
\newtheorem*{mondai*}{問題}
\newtheorem{wakunasimondai}{問題}
\newtheorem*{wakunasimondai*}{問題}
\tcbset{
  common/.style={
    enhanced,
    colframe=black, colback=white,
    coltitle=black, fonttitle=\bfseries, separator sign none,
    description color=black,
    sharp corners, frame hidden,
    boxsep=0mm, left=.9\zw, right=1\zw, middle=.5\zw,
    before upper={\parindent=1\zw},
    breakable
  },
  commonthm/.style={
    top=.5\zw, bottom=.5\zw, borderline={.1\zw}{0mm}{black,dotted}
  },
  commonproof/.style={
    top=0mm, bottom=0mm, borderline east={.5pt}{0mm}{black},
  }
}
\tcolorboxenvironment{thm}{common,commonthm}
\tcolorboxenvironment{lem}{common,commonthm}
\tcolorboxenvironment{prop}{common,commonthm}
\tcolorboxenvironment{cor}{common,commonthm}
\tcolorboxenvironment{conj}{common,commonthm}
\tcolorboxenvironment{dfn}{common,commonthm}
\tcolorboxenvironment{rem}{common,commonthm}
\tcolorboxenvironment{exm}{common,commonthm}
\tcolorboxenvironment{mondai}{common,commonthm}
\tcolorboxenvironment{mondai*}{common,commonthm}
% \tcolorboxenvironment{wakunasimondai}{common}


% proof style
% \makeatletter
% \renewenvironment{proof}[1][\proofname]{\par
%   \pushQED{\qed}%
%   \normalfont \topsep6\p@\@plus6\p@\relax
%   \trivlist
%   \item\relax
%         {%\itshape
%   #1。%\@addpunct{.}
%   }\hspace\labelsep\ignorespaces
% }{%
%   \popQED\endtrivlist\@endpefalse
% }
% \makeatother

\tcolorboxenvironment{proof}{common,commonproof}
\renewcommand{\proofname}{\bfseries 証明}


% makes table useful
\usepackage{multirow}
\usepackage{threeparttable}
\usepackage{booktabs}
\usepackage{csvsimple}
\usepackage{supertabular}


% hypertext marks
\usepackage{hyperref}


% cleveref
\usepackage{cleveref}
\crefname{appendix}{付録}{付録}
\crefname{equation}{式}{式}
\crefname{enumi}{}{}
\crefname{figure}{図}{図}
\crefname{section}{節}{節}
\crefname{table}{表}{表}
\crefname{listing}{ソースコード}{ソースコード}
\crefname{wakunasimondai}{問題}{問題}
\newcommand{\crefrangeconjunction}{\horizontalbar}
\newcommand{\crefpairconjunction}{、}
\newcommand{\crefmiddleconjunction}{、}
\newcommand{\creflastconjunction}{、}


% URL
\usepackage{url}


% BibLaTeX
\usepackage{biblatex}


% tategaki
% https://qiita.com/zr_tex8r/items/6f0b88c5838c42241457
\usepackage{lltjext}


% ipsum
\usepackage{lipsum}


% amsfonts
\usepackage{mathtools,amssymb}


% SI units
\usepackage{siunitx}
\sisetup{
  separate-uncertainty,
  per-mode=symbol,
  input-digits = 0123456789\pi,
  uncertainty-mode = separate,
  output-open-uncertainty = [,
  output-close-uncertainty = ],
  uncertainty-separator = \,,
  separate-uncertainty-units = bracket,
}


% multi columns
\usepackage{multicol}


% TikZ
\usepackage{tikz}
\usetikzlibrary{arrows.meta,petri,positioning}
\usepackage{standalone}
\usepackage{pgfplots}
\pgfplotsset{compat=1.18}
\usepackage{pgfplotstable}
\usepackage{shellesc}
\usepackage{gnuplottex}
\usepackage{gnuplot-lua-tikz}

% https://tex.stackexchange.com/questions/36147/how-do-i-get-the-gnuplottex-epstopdf-package-to-work-with-output-dir-and-au
\usepackage{etoolbox}
\let\nodirfigname\figname
\def\figname{./out/\nodirfigname}
\expandafter\patchcmd\csname\string\gnuplot\endcsname
  {\figname}{\nodirfigname}{}{}


% wrapfig
\usepackage{wrapfig}

% https://tex.stackexchange.com/questions/111393/too-much-space-around-wrap-figure
\setlength{\intextsep}{0pt}
\setlength{\columnsep}{0pt}


% subfigure
\usepackage{subcaption}


% define minipage environment with an unknown width
% use case: https://stackoverflow.com/questions/59551243/how-to-align-an-enumerated-list-in-latex
\usepackage{varwidth}
\newenvironment{centered}{
  \begin{center}
    \begin{varwidth}{\textwidth}
}{
    \end{varwidth}
  \end{center}
}


% multicolumn itemize
% reduce margin of the environment
\newenvironment{multicolsitemize}[1]{
  \vspace*{-.5\multicolsep}
  \begin{multicols}{#1}
    \begin{itemize}
}{
    \end{itemize}
  \end{multicols}
  \vspace*{-.5\multicolsep}
}


% tweaking vertical space
\newcommand{\tweak}[1]{\vspace{#1\zw}}


% kintou
\newcommand{\kintou}[2]{\texorpdfstring{\kintouwidth{#1\zw}{#2}}{#2}}
% 美文書作成入門に載ってるもの
% fancyhdr の中に入るとうまくいかないので使えないのてとりあえず下で我慢
% \newcommand{\kintouwidth}[2]{\leavevmode\hbox to #1{%
%   \ltjsetparameter{kanjiskip=0pt plus 1fill minus 1fill}%
%   \ltjsetparameter{xkanjiskip=\ltjgetparameter{kanjiskip}}%
%   #2}}
% \makeatletter
\newcommand{\kintouwidth}[2]{\makebox[#1][s]{#2}}


% mask
% https://tex.stackexchange.com/questions/548426/how-to-make-text-appear-in-the-middle-of-a-phantom-in-math-mode
\usepackage{calc}
\newcommand*{\mask}[2]{%
  \makebox[\widthof{\(#1\)}]{\(#2\)}%
}


% displaystyle
\newcommand{\sfrac}[2]{\genfrac{}{}{}{0}{\,#1\,}{\,#2\,}}
\newcommand{\dinf}{\displaystyle\inf}
\newcommand{\dint}{\displaystyle\int}
\newcommand{\dlim}{\displaystyle\lim}
\newcommand{\dmax}{\displaystyle\max}
\newcommand{\dmin}{\displaystyle\min}
\newcommand{\dprod}{\displaystyle\prod}
\newcommand{\dsum}{\displaystyle\sum}
\newcommand{\dsup}{\displaystyle\sup}


% horizontal bar
\newcommand{\horizontalbar}{\symbol{8213}}


% roman abbrev
\newcommand{\GL}{\text{GL}}
\newcommand{\M}{\text{M}}
\newcommand{\SL}{\text{SL}}


% op
\DeclareMathOperator{\ch}{char}
\DeclareMathOperator{\id}{id}


% set-builder
% https://tex.stackexchange.com/questions/253077/how-do-you-create-a-set-in-latex
\DeclarePairedDelimiterX\set[1]\lbrace\rbrace{\def\given{\;\delimsize\vert\;}#1}


% round function
% https://tex.stackexchange.com/questions/433101/rounding-to-nearest-integer-symbol-in-latex
\DeclarePairedDelimiter{\nint}{\lfloor}{\rceil}


% array
% https://github.com/texjporg/jsclasses/issues/61
\newcommand{\nbl}{\narrowbaselines}
\everymath=\expandafter{\the\everymath \narrowbaselines}


% lineskip
\setlength{\lineskiplimit}{4pt}
\setlength{\lineskip}{4pt}


% generate abbreviations for mathbb, bm, mathcal, mathrsfs, and mathfrak
\usepackage{bm}
\usepackage{mathrsfs}
\directlua{
  local command_mapping = {
    ["bb"] = "mathbb",
    ["bm"] = "bm",
    ["cal"] = "mathcal",
    ["rm"] = "mathrm",
    ["scr"] = "mathscr",
    ["frk"] = "mathfrak",
  }

  for command, replacement in pairs(command_mapping) do
    for charCode = string.byte('a'), string.byte('z') do
      local char = command == "bm" and string.char(charCode) or string.char(charCode):upper()
      tex.print("\\newcommand{\\" .. command .. char:lower() .. "}{\\" .. replacement .. "{" .. char .. "}}")
    end
  end
}


% https://tex.stackexchange.com/a/548869
\usepackage{ifthen}
\usepackage{currfile-abspath}
\getabspath{\jobname.tex}
\ifthenelse{\equal{\theabsdir}{\thepwd}}% using ifthen package
%\ifdefstrequal{\theabsdir}{\thepwd}% using etoolbox package
    {\PassOptionsToPackage{outputdir={out}}{minted}}{\PassOptionsToPackage{outputdir={\theabsdir/out}}{minted}}


% see for setup
% https://zenn.dev/nac_39/scraps/25ffb7d0747a9d
% \directlua の前に置くと何故かエラーを吐く
\usepackage{minted}
\definecolor{bg}{rgb}{0.97,0.97,0.97}
\setminted{
  baselinestretch=.75,
  bgcolor=bg,
  breaklines,
  breakanywhere,
  gobble=2,
}
\renewcommand{\listingscaption}{ソースコード}
\usepackage{caption}
\captionsetup[listing]{labelsep=quad}
\newenvironment{longlisting}{\captionsetup{type=listing}}{}

% prevent caption 1zw sep overwritten
% (maybe overwritten by `subcaption` package?)
% @see texmf-dist/tex/luatex/luatexja/ltjsarticle.cls
\makeatletter
\long\def\@makecaption#1#2{{\small
  \advance\leftskip .0628\linewidth
  \advance\rightskip .0628\linewidth
  \vskip\abovecaptionskip
  \sbox\@tempboxa{#1{\hskip1\zw}#2}%
  \ifdim \wd\@tempboxa <\hsize \centering \fi
  #1{\hskip1\zw}#2\par
  \vskip\belowcaptionskip}}
\makeatother
